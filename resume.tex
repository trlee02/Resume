\documentclass[a4paper]{MagicalCV}

% Removing page numbers
\usepackage{fancyhdr}
\pagestyle{fancy}
\fancyhf{}

% Change the geometry
\geometry{left=1.4cm, top=.8cm, right=1.4cm, bottom=1.8cm, footskip=.5cm}

% Defining your colors
\definecolor{VividPurple}{HTML}{3E0097}
\definecolor{SlateGrey}{HTML}{2E2E2E}
\definecolor{LightGrey}{HTML}{666666}
\colorlet{heading}{VividPurple}
\colorlet{accent}{VividPurple}
\colorlet{emphasis}{SlateGrey}
\colorlet{body}{LightGrey}

% Color for highlights
% Awesome Colors: awesome-emerald, awesome-skyblue, awesome-red, awesome-pink, awesome-orange
%                 awesome-nephritis, awesome-concrete, awesome-darknight
\colorlet{awesome}{awesome-purple}

% Set false if you don't want to highlight section with awesome color
\setbool{acvSectionColorHighlight}{true}

% If you would like to change the social information separator from a pipe (|) to something else
\renewcommand{\acvHeaderSocialSep}{\quad\textbar\quad}

% start document
\begin{document}

% Last update time
\lastupdated

% Title name
\namesection{}{Tristan Lee}{
\phone{+1(778)363-7904} \email{\href{mailto:tristan.rene.lee@gmail.com}{tristan.rene.lee@gmail.com}}}

% Column one
\begin{minipage}[t]{0.25\textwidth} 

% Skills
\cvsection{Skills}

\subsection{Electrical}
Altium \textbullet{} Oscilliscope \textbullet{} Eagle \textbullet{} Soldering 
\sectionsep
\subsection{Programming}
Python \textbullet{} Java \textbullet{} C/C++ \textbullet{} Git \textbullet{} Linux

\sectionsep
\subsection{Mechanical}
OnShape \textbullet{} SolidWorks \textbullet{} Fusion 360 \textbullet{} 3D Printing

% Education
\cvsection{Education} 

\subsection{U. of British Columbia}
\descript{Engineering Physics}
\cvevent{Grad. 2026}{Vancouver, BC} 
\vspace{\topsep} % Hacky fix for awkward extra vertical space
Third Year, 88.8\% Avg.

% Links
\cvsection{Links}

\linkedin{Linkedin} \href{https://www.linkedin.com/in/tristanrlee/}{\bf linkedin.com/in/tristanrlee/}\\
\github{GitHub} \href{https://github.com/trlee02/}{\bf trlee02} 
\sectionsep

% Awards
\cvsection{Awards} 

\subsection{Presidential Scholars}
\descript{University of British Columbia}
\vspace{\topsep} % Hacky fix for awkward extra vertical space
Awarded to accomplished Canadian students.
\sectionsep
\subsection{Tuum Est Experiential}
\descript{University of British Columbia}
\vspace{\topsep} % Hacky fix for awkward extra vertical space
Awarded to students with excellent academic standing and strong personal profiles.
\sectionsep

\subsection{Trek Excellence}
\descript{University of British Columbia}
\vspace{\topsep} % Hacky fix for awkward extra vertical space
Awarded to top 5\% of UBC undergraduate students.
\sectionsep

% Interests
\cvsection{Interests}\\
Robotics \\
Machine learning  \\
Rocketry\\
Downhill Skiing \\
Mountain Biking \\
Surfing \\
Hiking\\
Powerlifting \\

\sectionsep

% Column two
\end{minipage} 
\hfill
\begin{minipage}[t]{0.72\textwidth} 

% Experience
\cvsection{Technical Experience}

\runsubsection{Manufacting Test Engineer} \\
\descript{Enersys - Alpha Technologies}
\cvevent{Jan. 2022 – May 2022}{Vancouver, BC} 
\vspace{\topsep} % Hacky fix for awkward extra vertical space
\begin{tightemize}
\item Assembled 5 PCB test stands, validated LabVIEW signal tests to specific pins using an oscilliscope, troubleshot and repaired connections and tests to ensure proper performance.
\item Created Python and LabVIEW software to enable data collection and PDF conversion for PCB tests, then implemented the software into 10 different test stands. 
\item Constructed circuit schematics in Altium Designer and wrote test scripts in LabVIEW for PCB test stands, as well as identified test points in Altium for a variety of DC-DC converters.  
\end{tightemize}
% \sectionsep

% Recent Projects
\cvsection{Project Experience}

\runsubsection{\href{https://projectlab.engphys.ubc.ca/enph-253/}{Engineering Physics Robot Competition}} \\
\descript{University of British Columbia}
\cvevent{May 2022 – Aug 2022}{Vancouver, BC} 
\begin{tightemize}
    \item Collaborated with a group of 4 to design and manufacture an item retrieval robot that navigated a course using line following and 10kHz IR sensing, acheiving 4th place.
    \item Designed and constructed over 10 circuits including power distribution for motors and sensors, DC motor drivers, stepper motor drivers, and microcontroller pin distribution.
    \item Troubleshot and tested many circuits constructed with my teammates, to ensure the presence of desired signals using an oscilliscope.
    \item Integrated firmware into C++ statemachine using PlatformIO to control a linearly translating robot arm and 2 claws, as well as sense retrievable items using sonar sensors. 
    \item Created CAD designs for the chassis and claw sections of our robot using OnShape.  
\end{tightemize}
\sectionsep

\runsubsection{Engineering Physics Machine Learning Competiton} \\
\descript{University of British Columbia}
\cvevent{Sep 2022 - Dec 2022}{Vancouver, BC} 
\begin{tightemize}
        \item Worked in a group of 2 to design and create state machine architecture to control a robot using ROS Noetic on a simulated course in Gazebo.
        \item Implemented OpenCV in Python to capture images of license plates inside a simulated enviroment and identify characters using a convolution neural network.
        \item Setup a directory structure and Gazebo enviroment in Linux needed to collect data for convolution neural network training for robot self driving.
        % \item Implemented a QLearning algorithm in Gym Gazebo and explored other types of reinforcement learning.
\end{tightemize}
\sectionsep

\runsubsection{UBC Rocket Avionics} \\
\descript{University of British Columbia}
\cvevent{Oct 2022 - Present}{Vancouver, BC} 
\begin{itemize}
    \item Designed half-bridge e-match ignition PCB in Altium designer, as a part of a stackable, modular flight computer.
    \item Learning manufacturing and testing methods for our teams PCBs.   
    \item Currently collaborating with a team of six to begin testing and manufacturing of flight computers.
\end{itemize} 
\sectionsep


% \runsubsection{Email Interaction Analytics Program} \\
% \descript{University of British Columbia}
% \cvevent{Nov 2021 – Dec 2021}{Vancouver, BC} 
% \begin{itemize}
%     \item 
% \end{itemize} 

\end{minipage} 
\end{document}  